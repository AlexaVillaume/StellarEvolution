\documentclass[12pt]{article}
 \usepackage{amsmath}
 \usepackage{mathtools}
 \usepackage{accents}
\usepackage{hyperref}


 \newcommand*{\dt}[1]{%
  \accentset{\mbox{\large\bfseries .}}{#1}}
  
 \begin{document}
 
 \title{Problem Set 3}
\author{Alexa Villaume\\ 
Stellar Evolution } 
 
\maketitle
 
\noindent \textbf{Problem 1 Polytrope whatnots}\\
\noindent \textbf{(a)} Derive the relation between $M$ and $R$ for polytropes. How does it depend on $n$? Discuss what kinds of objects might be close to $n=0, 1, 1.5$ polytropes and how their radii change by adding mass. \\

\noindent Let's begin with what we know about $M$,

\begin{equation}
dm = \int_0^R 4\pi r^2 \rho dr,
\end{equation}

\noindent and what we know about some of these values for polytropes,

\begin{equation}
r = r_n \xi, dr = r_n d\xi,
\end{equation}

\begin{equation}
\rho = \rho_c \Theta^n.
\end{equation}

\noindent Which we can plug back in to the $dm$ equation,

\begin{equation}
dm = \int_0^\xi 4\pi\left( r_n \xi \right)^2 \left( \rho_c \Theta_n \right) \left( r_n d\xi \right).
\end{equation}

\noindent Taking the constants out of the integral,

\begin{equation}
dm = 4\pi \rho_c r_n^2 \int_0^\xi \xi^2 \Theta_n d\xi.
\end{equation}

\noindent We can substitute the Lane-Emden equation,

\begin{equation}
-\Theta_n^n = \frac{1}{\xi^2}\frac{d}{d\xi}\left(\xi^2 \frac{d\Theta_n}{d\xi} \right),
\end{equation}

\noindent into the integral to get,

\begin{equation}
dm = -4\pi \rho_c r_n^2 \int_0^\xi \xi^2 \left[ \frac{1}{\xi^2} \frac{d}{d\xi} \left( \xi^2 \frac{d\Theta_n}{d\xi} \right) \right] d\xi.
\end{equation}

\noindent Now, we just start simplifying,

\begin{equation}
dm = -4\pi \rho_c r_n^2 \int_0^\xi \frac{d}{d\xi} \left( \xi^2 \frac{d\Theta_n}{d\xi} \right) d\xi.
\end{equation}

\noindent We can take the integral,

\begin{equation}
dm = -4\pi \rho_c r_n^2 \left( \xi^2 \frac{d\Theta_n}{d\xi} \right).
\end{equation}

\noindent Now we can start substituting in for things we know. Like,

\begin{equation}
r_n = \sqrt{\frac{\left( n + 1 \right)P_c}{4\pi\rho_c^2 G}}, r_n^3 = \left[ \frac{\left( n + 1 \right)P_c}{4\pi\rho_c^2 G} \right]^{2/3} \frac{1}{\rho_c^3}
\end{equation}

\begin{equation}
dm = -4\pi \left[ \frac{\left( n + 1 \right) P_c}{4\pi\rho_c^2 G} \right]^{2/3} \frac{1}{\rho_c^3} \rho_c \xi^2 \frac{d\Theta_n}{d\xi}
\end{equation}

\noindent Since $P_c = K \rho_c^{1 =1/n}$,

\begin{equation}
dm = -4\pi \left[ \frac{\left( n + 1 \right)}{4\pi G} \right]^{2/3} \left(K \rho_c^{1+ 1/n}\right)^{2/3}  \frac{1}{\rho_c^2}  \xi^2 \frac{d\Theta_n}{d\xi}
\end{equation}

\noindent Where we can simplify the $\rho_c$ term,


\begin{equation}
dm = -4\pi \left[ \frac{\left( n + 1 \right)K }{4\pi G} \right]^{2/3} \rho_c^{\frac{3-n}{2n}} \xi^2 \frac{d\Theta_n}{d\xi}
\end{equation}

\noindent And we'll solve for $\rho_c$, 

\begin{equation}
\rho_c = \left[ \frac{M}{4\pi} \left[\xi^2\left( \frac{-d\Theta_n}{d\xi}\right) \right]^{-1} \left[ \frac{4\pi G}{\left( n + 1 \right) K} \right]^{3/2} \right]^{\frac{2n}{3-n}}
\end{equation}

\noindent $r=r_n\xi$ can be written in terms with our derived $\rho_c$ value,

\begin{equation}
r = \left[ \frac{\left( n + 1 \right)K}{4\pi G} \right]^{1/2}\left[ \frac{M}{4\pi} \left[\xi^2\left( \frac{-d\Theta_n}{d\xi}\right) \right]^{-1} \left[ \frac{4\pi G}{\left( n + 1 \right) K} \right]^{3/2} \right]^{\frac{2n}{3-n}}
\end{equation}

\noindent Since that is a big, awful mess I'm going to ignore everything but $R$ and $M$ and say,

\begin{equation}
R \propto M^{\frac{1-n}{3-n}}
\end{equation}

\begin{itemize}
\item $n=0$, as the radius increases, mass increases. Examples of this would be small planets, moons, etc...
\item $n=1$, radius does not depend on mass. Like gas giants and brown dwarfs.
\item $n=3/2$, and the radius decreases, mass increases so a very centrally dense object like a white dwarf.
\end{itemize}

\noindent \textbf{(b)} For a polytropic star of mass $M$, radius $R$, and polytropic index $n$, find an expression for the gravitational potential energy. \\

\noindent Let's start with equation 19.37 from Kippenhan,

\begin{equation}
E_g = \frac{1}{2}\int^M_0 \Phi dm - \frac{1}{2}\frac{GM^2}{R}
\end{equation}
 
\noindent We know that $P=K\rho^{1+\frac{1}{n}}$, $K = \frac{P}{\rho}^{1 +\frac{1}{n}}$ so we can get,

\begin{equation}
\Phi = - K\left( n + 1 \right) \rho^{1/}
\end{equation}

\begin{equation}
\Phi = - \frac{P\left( n + 1 \right) \rho^{1/n}}{\rho^{1 + 1/n}}
\end{equation}

\begin{equation}
\Phi = -\frac{P}{\rho}\left( n + 1 \right)
\end{equation}

\noindent So equation 19.37 becomes,

\begin{equation}
E_g = -\frac{1}{2} \frac{GM^2}{R} - \frac{\left( n +1 \right)}{2}\int_0^M \frac{P}{\rho}dm.
\end{equation}

\noindent We also know,

\begin{equation}
\int_0^M \frac{Gm}{r}dm = 3\int_0^M \frac{P}{\rho}dm
\end{equation}

\noindent and, 

\begin{equation}
E_g = -\int_0^M \frac{Gm}{r}dm.
\end{equation}

\noindent which means,

\begin{equation}
\frac{1}{3}\int_0^M\frac{Gm}{r} = \int_0^M \frac{P}{\rho}dm
\end{equation}

\begin{equation}
\int_0^M \frac{P}{\rho}dm = -\frac{1}{3}E_g
\end{equation}

\noindent which we can put back into equation 19.37,

\begin{equation}
E_g = -\frac{1}{2}\frac{GM^2}{R} + \frac{1}{6}\left(n + 1 \right)E_g.
\end{equation}

\noindent And now it's just a matter of solving for $E_g$

\begin{equation}
E_g\left( 1 - \frac{n+1}{6} \right) = -\frac{1}{2}\frac{GM^2}{R}
\end{equation}

\begin{equation}
E_g = -\frac{GM^2}{2R}\left( \frac{1}{1 - \frac{\left( n + 1\right)}{6}} \right)
\end{equation}

\noindent So let's unpack that second term,

\begin{equation}
\left( \frac{1}{1 - \frac{\left( n + 1\right)}{6}} \right) = \left( \frac{1}{\frac{6}{6} - \frac{\left( n + 1\right)}{6}} \right) = \left( \frac{1}{\frac{6 - \left(n + 1 \right)}{6}}\right) = \frac{6}{5 -n}
\end{equation}

\noindent Which we can put back into the equation to get,

\begin{equation}
E_g = -\frac{GM^2}{2R}\left(   \frac{6}{5 -n} \right) =  -\frac{GM^2}{R}\left(   \frac{3}{5 -n} \right)
\end{equation}
\\
\noindent \textbf{Problem 2} As mentioned in class, there $are$ physical objects with a polytrope index close to $n=1$. They are giant planets and brown dwarfs. In class we wrote that $\Theta_1 = \left( sin \xi \right)/\xi$. Let's work with Jupiter, which as a radius of 70,000 km (1 $R_j$) and a mass of $1.9 \times 10^{30}$ g (1 $M_j$).\\

 
 \noindent \textbf{(a)} Show that  $\Theta_1 = \left( sin \xi \right)/\xi$ and find $\xi_1$. \\
 
 \noindent Let's begin with the Lane-Emden equation,

\begin{equation}
-\Theta_n^n = \frac{1}{\xi^2}\frac{d}{d\xi}\left ( \xi^2 \frac{d\Theta_n}{d\xi}\right).
\end{equation}

\noindent Now we'll re-arrange and acknowledge that $n=1$,

\begin{equation}
-\xi^2\Theta_n = \frac{d}{d\xi}\left ( \xi^2 \frac{d\Theta_n}{d\xi}\right).
\end{equation}

\noindent Let's assume the solution and substitute it into the above equation, 
\begin{equation}
-\xi^2 \left( sin \xi \right)/\xi = \frac{d}{d\xi}\left ( \xi^2 \frac{d\left( sin \xi \right)/\xi}{d\xi}\right),
\end{equation}

\noindent and then use the Chain Rule to start simplifying, 

\begin{equation}
-\xi^2 \left( sin \xi \right)/\xi =  \frac{d}{d\xi}\left ( \frac{\xi\mathrm{cos}\xi - \mathrm{sin}\xi}{\xi}\right),
\end{equation}

\noindent Which we can simplify further to get,

\begin{equation}
-\xi \left( sin \xi \right)/\xi = -\xi \left( sin \xi \right)/\xi.
\end{equation}

\noindent To get $\xi_1$ we just need to know what value brings $\Theta\left(\xi_1\right) = 0$, which is $\pi$.  \\

\noindent \textbf{(b)} Using what is given above, solve for $K$. \\

\begin{equation}
R = \xi R_{n} = \xi \sqrt{\frac{\left( n + 1\right)P_c}{4\pi G\rho_c^2}},
\end{equation}

\begin{equation}
P_c = K \rho_c^{1 + \frac{1}{n}}
\end{equation}
  
\begin{equation}
R = \xi \sqrt{\frac{\left( n + 1\right)K  \rho_c^{1 + \frac{1}{n}}}{4\pi G\rho_c^2}},
\end{equation}

\noindent Since $n=1$

\begin{equation}
R = \xi \sqrt{\frac{\left( n + 1\right)K }{4\pi G}},
\end{equation}
  
\noindent Solving for $K$, 

\begin{equation}
K = \frac{4\pi G R^2}{\xi^2 \left( n+1\right)}
\end{equation}
  
\noindent We can plug in values, to get, 

\begin{equation}
K = \frac{2 G R_j^2}{\pi } = 2.082 \times 10^{12}
\end{equation}

\noindent \textbf{(c)} Now solve for the central density, $\rho_c$. \\

\begin{equation}
\frac{\rho_c}{\left< \rho \right>} = \frac{1}{3}\left( \frac{\xi}{-\Theta^{'}_n}\right)_{\xi_1}
\end{equation}

\begin{equation}
\frac{\rho_c}{\left< \rho \right>} = \frac{1}{3}\left( \frac{\xi_1}{-cos \xi_1}\right)_{\xi_1} = \frac{\pi}{3}.
\end{equation}

\begin{equation}
\rho_c = \frac{\pi}{3}\frac{M}{V} = \frac{\pi}{4}\frac{M_j}{R_j^3} =  4.35
\end{equation}

\noindent \textbf{(d)} In cgs units, write out the density as a function of $r$, $\rho\left( r\right)$.\\

\begin{equation}
\rho \left( r \right) = \Theta^n\left( r \right) \rho_c
\end{equation}

\noindent We know from 2b that $\Theta_1 = \left( sin \xi \right)/\xi$ and from the scale height relation that $\xi = r/r_n$ so, 

\begin{equation}
\Theta_1 = \frac{\mathrm{sin}\left( r/r_n \right)}{r/r_n}
\end{equation}

\noindent

\begin{equation}
r_n = \sqrt{\frac{\left(n+1\right)P_c}{4\pi \rho_c^2 G}}
\end{equation}

\noindent Since for an $n=1$ polytrope $P_c = K\rho^2$ we can simplify the expression for the scale-height, 

\begin{equation}
r_n = \sqrt{\frac{K}{2\pi G}}
\end{equation}

\noindent From 2b we know $K$,

\begin{equation}
r_n = \sqrt{\frac{2GR_j^2}{\pi}\frac{1}{2\pi G}} = \frac{R_j}{\pi}.
\end{equation}

\noindent Which we can substitute back into the expression for $\Theta_1$,

\begin{equation}
\Theta_1 = \frac{\mathrm{sin}\left( r\pi/R_j \right)}{r\pi/R_j}
\end{equation}

\noindent So, then, $p\left (r\right)$ is given by,

\begin{equation}
p\left (r\right) = \Theta\left(r \right)\rho_c.
\end{equation}

\noindent \textbf{(e)} Plot pressure vs. density, and density vs. radius for the polytrope Jupiter, and compare those to a ``real'' Jupiter model at \url{http://www.ucolick.org/~jfortney/classes/Jup_Guillot99.dat}.\\

\noindent Figures below. \\

\noindent \textbf{(f)} Using what you know, solve for the radius of a 50 $M_j$, $n=1$ polytrope. \\

\noindent We know from 1a that for a $n=1$ polytrope the radius does not change as a function of mass. Therefore, the radius of a 50$M_j$ polytrope will be the same as a 1 $M_j$ polytrope.\\

\noindent \textbf{(g)} Let's assume Jupiter, by number, is 0.9 H and 0.1 He, with the H fully ionized and the He not ionized. Find $\mu$ for Jupiter. \\

\noindent Since Helium is not ionized we begin with the following equation,

\begin{equation}
\mu = \left ( \Sigma_i  \frac{X_i \left( 1 + Z_i\right) }{\mu_i} \right)^{-1}.
\end{equation}

\noindent Where $X_i$ is the mass fraction of a given species, $Z_i$ is the number of electrons, and $\mu_i$ is the molecular weight. Since Hydrogen is ionized $Z_H$ = 1 and since Helium is not ionized $Z_{He}$ = 0 and $\mu_H \approx$  1 and $\mu_{He} \approx$  4. We're given the fraction by number which we have to turn into a mass fraction. This is done by, \\

\begin{equation}
\mathrm{Mass}_H = \frac{0.9M_H}{0.9M_H + 0.1M_{He}} = 0.69,
\end{equation}

and,

\begin{equation}
\mathrm{Mass}_{He} = \frac{0.1M_{He}}{0.9M_H + 0.1M_{He}} = 0.31.
\end{equation}

\noindent Putting everything back into the original equation to get, $\mu \approx$ 0.68.\\

\noindent \textbf{(h)} Assuming that the ideal gas law hold, plot the pressure vs. temperature for real Jupiter and the polytrope. Do you think that the ideal gas law holds?\\
 
\noindent Figure below. Given the large difference in the real model and my solution, I would say that the ideal gas law \textit{doesn't} hold.\\

\noindent \textbf{Problem 3} Program the subroutines ``load1'' and ``load2'' to use in ``shootf''. Note that the central pressure $P_c$, the central temperature $T_c$, and the total luminosity $L$, and the outer radius $R$ are parameters that you guess in advance. Stellar $M$ and $\mu$ are chosen. Recall that helpful surface conditions are that $L_{tot} = 4\pi R^2 \sigma T_{eff}^4$ and $\kappa P = 2g/3$. \\

\noindent Program can be found at \url{https://github.com/AlexaVillaume/StellarEvolution/blob/master/ProblemSet_3/shootf.py} \\

\noindent \textbf{Problem 4} Program the subroutine ``derives''. For an independent variable $x$ (the mass coordinate) and the four dependent variables $y_i$ at that point, the subroutine should calculate $dy_i/dx$. \\

\noindent Program can be found at \url{https://github.com/AlexaVillaume/StellarEvolution/blob/master/ProblemSet_3/shootf.py} \\

\begin{figure}[b]
  \vspace{-30pt}
  \begin{center}
    \includegraphics[width=1\textwidth]{plot_2_e.pdf}
  \end{center}
  \vspace{-20pt}
  \vspace{-15pt}
\end{figure}  % figures there
 
 
 \begin{figure}[t]
  \vspace{-30pt}
  \begin{center}
    \includegraphics[width=0.75\textwidth]{plot_2_h.pdf}
  \end{center}
  \vspace{-20pt}
  \vspace{-15pt}
\end{figure}  % figures there
 
\end{document}